\documentclass{article}
\usepackage[hidelinks]{hyperref}
\usepackage{csquotes}
\usepackage[vmargin=25mm, hmargin=20mm]{geometry}
\usepackage{lipsum}
\usepackage{xcolor}
% \usepackage{graphicx}
% \usepackage{float}
% \usepackage{pgfplots}
% \usepgfplotslibrary{colorbrewer}
% \pgfplotsset{compat = 1.18} 
% \usetikzlibrary{pgfplots.statistics, pgfplots.colorbrewer} 
% \usepackage{pgfplotstable}
\usepackage{listings}
\lstset{
    backgroundcolor=\color[RGB]{240, 240, 240},   
    basicstyle=\ttfamily\footnotesize,
    breakatwhitespace=false,
    breaklines=true,
    keepspaces=true,
    numbers=left,
    numbersep=5pt,
    showspaces=false,
    showstringspaces=false,
    showtabs=false,
    tabsize=4,
    postbreak=\mbox{\textcolor{red}{$\hookrightarrow$}\space},
    aboveskip=10pt
}
\usepackage{listings-rust}

\usepackage[
    backend=biber,
    sorting=none,
    style=ieee,
    urldate=long,
    maxcitenames=2,
    mincitenames=1
]{biblatex}
\addbibresource{sources.bib}
\renewcommand*{\bibfont}{\footnotesize}

\usepackage{multicol}
\setlength{\columnsep}{13mm}

% \usepackage{hypcap}
% \usepackage{caption}
% \captionsetup{
%     justification=centerlast,
%     font=small,
%     labelfont=sc,
%     margin=5pt,
%     belowskip=15pt
% }
% \DeclareCaptionType{listing}[Listing][List of Listings]

\title{%
    \vspace{50px}%
    \Huge A Stateful Fuzzer for the TCP/IP Stack of the Real-Time Operating System Zephyr
    \vspace{250px}%
}

\author{%
    Valentin Huber\vspace{5px}\\%
    \small at \href{https://www.cydcampus.admin.ch/}{Cyber Defence Campus}\\%
    \small and \href{https://www.zhaw.ch/en/engineering/institutes-centres/init/}{Institute of Computer Science at ZHAW}\\%
    \small \href{mailto://contact@valentinhuber.me}{contact@valentinhuber.me}%
    \vspace{10px}
}

\date{\today}

\DeclareFieldFormat*{citetitle}{\textit{#1}}
\hfuzz=50px
\hbadness=10000
\newcommand{\code}[2][]{\lstinline[language=#1, breaklines=false, basicstyle=\ttfamily\normalsize]{#2}}
\let\savedCite=\cite
\renewcommand{\cite}{\unskip~\savedCite}
\renewcommand{\arraystretch}{1.5} % Increase table padding


\begin{document}
\pagenumbering{gobble}
\maketitle

\clearpage\newpage
\begin{center}
  \begin{minipage}{0.8\textwidth}
    \vspace{70px}

    \begin{abstract}
      \lipsum[1]\lipsum[2]\lipsum[3]
    \end{abstract}
  \end{minipage}

  \vspace{70px}

  \begin{minipage}{0.7\textwidth}
    \textbf{Keywords}: Software Testing, Fuzzing, Stateful Fuzzing, Zephyr, LibAFL.
  \end{minipage}
\end{center}

\clearpage\newpage
\tableofcontents
\vspace{30px}
\pagenumbering{arabic}
\clearpage\newpage

\begin{multicols}{2}

  \section{Introduction}

  \begin{itemize}
    \item Fuzzing is a widely used strategy to find software defects
    \item Overview: How does fuzzing work?
  \end{itemize}

  \subsection{Zephyr}

  \begin{itemize}
    \item Zephyr is an open-source RTOS developed by the Linux Foundation.
    \item Overview: Where is Zephyr used?
    \item The TCP/IP stack in Zephyr is particularly critical for the security of many products Zephyr is used in.
  \end{itemize}

  \subsection{Fuzzing Network Stacks Is Hard}

  \subsubsection{Deep integration with OS}
  \begin{itemize}
    \item Environment/Hardware modelling necessary
    \item Usually, whole system needs to be run, which is a performance issue
    \item Instrumentation (Coverage, Sanitizers) is hard
  \end{itemize}

  \subsubsection{Network packets are highly structured}
  \begin{itemize}
    \item Dependencies between data (length fields, checksums, etc.)
    \item Dependencies between packages (state in TCP)
  \end{itemize}

  \subsubsection{TCP Stacks Have an Internal State}
  \begin{itemize}
    \item A certain packet results in different behavior depending on previous activity
  \end{itemize}


  \subsection{Quotes, TODO: Integrate}

  \citeauthor{StatefulReview} introduce a taxonomy of stateful fuzzing in which they define a stateful system as \textquote{a system that takes a sequence of messages as input, producing outputs along the way,and where each input may result in an internal state change}\cite{StatefulReview}.

  \textquote{To avoid confusion, we reserve the term message or input message for the individual input that the System Under Test (SUT) consumes at each step and the term trace for a sequence of such messages that make up the entire input.}\cite{StatefulReview}

  \textquote{The input language of a stateful system consists of two levels: (1) the language of the individual messages, which we will refer to as the message format, and (2) the language of traces, built on top of that. A description or specification of such an input language will usually come in two parts, one for each of the levels: for example, a context-free grammar for the message format and a finite state machine describing sequences of these messages. We will call the latter the state model or, if it is described as a state machine, the protocol state machine.}\cite{StatefulReview}

  \textquote{the internal state changes increase the state space that we try to explore: it may be hard for a fuzzer to reach 'deeper' states. Indeed, fuzzing stateful systems is listed as one of the challenges in fuzzing by \citeauthor{ChallengesAndReflections}\cite{ChallengesAndReflections}}\cite{StatefulReview}.

  \subsubsection{TCP/IP}
  Checksums such as in TCP/IP are challenging.\cite{StateOfTheArt} Approaches to deal with this include removing the program sections from the target program\cite{TFuzz}.

  \subsection{Contributions of this Project}
  A fuzzer that targets the TCP/IP stack of Zephyr, which
  \begin{itemize}
    \item employs a framework that allows running Zephyr as a native executable, removing the necessity of an emulation or translation layer,
    \item introduces a custom ethernet driver based on shared memory to make fuzzing parallelizable and performant,
    \item guides mutation based on coverage information and a heuristic to infer the state of the TCP state machine based on flags in the TCP header of responses from the server
  \end{itemize}
  This project further
  \begin{itemize}
    \item evaluates how different ways of using the feedback described above influence fuzzer performance,
    \item evaluates different ways of modelling the packets passed to the fuzzer,
    \item evaluates different approaches to mutating those, and
    \item introduces various performance optimizations to LibAFL
  \end{itemize}

  \section{Related Works}
  \begin{itemize}
    \item Focus on Network Protocol Fuzzing
  \end{itemize}

  \subsection{Protocol Fuzzing}
  \subsubsection{Observing State}
  \begin{itemize}
    \item Manual annotation
    \item Automatic annotation
    \item Other greybox approaches
    \item Heuristics
  \end{itemize}

  \subsubsection{Use of State Information}
  \begin{itemize}
    \item Feedback/Maximization
    \item Scheduling
  \end{itemize}

  \subsubsection{Input Modelling and Mutation}


  \subsection{Snapshotting}


  \subsection{Data, TODO: Integrate}
  \begin{itemize}
    \item \citetitle{StatefulGreybox}\cite{StatefulGreybox}: \textquote{In this paper, we argue that protocols are often explicitly encoded using state variables that are assigned and compared to named constants […] More specifically, using pattern matching, we identify state variables using enumerated types (enums). An enumerated type is a group of named constants that specifies all possible values for a variable of that type. Our instrumentation injects a call to our runtime at every program location where a state variable is assigned to a new value. Our runtime efficiently constructs the state transition tree (STT). The STT captures the sequence of values assigned to state variables across all fuzzer-generated input sequences, and as a global data structure, it is shared with the fuzzer.}\cite{StatefulGreybox} Built on LibFuzzer
    \item \citetitle{StateAFL}\cite{StateAFL}: compile-time probes observing memory allocation and I/O operations; state inference based on fuzzy hashing of long-lived memory areas.
    \item \citetitle{Ijon}\cite{Ijon}: Manual annotations of code, to manually add entries to an AFL-style map (set/inc at calculated offset), include state information (variable values) in how edge coverage is calculated, and store the max value a certain variable reaches during execution for the fuzzer to then maximize.
    \item \citetitle{SandPuppy}\cite{SandPuppy}: Ijon\cite{Ijon}, but automatic (initial run capturing variable-value traces, analyze along with source code, add Ijon-style instrumentation, repeat during fuzzing)
    \item \citetitle{INVSCOV}\cite{INVSCOV}: run for 24 hours, record variable values and relationships between them, then add a feedback that rewards when the generated assertions are violated
    \item \citetitle{Ankou}\cite{Ankou}: take combination of executed branches into consideration, reduce to manageable adaptive fitness function
    \item \citetitle{FuzzFactory}\cite{FuzzFactory}: framework to add custom feedbacks like number of basic blocks executed, amount of memory allocated, etc.
    \item \citetitle{ParmeSan}\cite{ParmeSan}: Use sanitizers checks as fuzzing targets
    \item \citetitle{DDFuzz}\cite{DDFuzz}: Use execution of new data dependencies as feedback
    \item \citetitle{StateFuzz}\cite{StateFuzz}: Find state variables (long-lived, can be updated by users, change control flow or memory access) using static analysis, use that to guide fuzzing (new coverage, new value-range, new extreme value). (Talk: Good Example of why coverage-guided alone is insufficient). Check value ranges instead of all values (static symbex!). 4-digit number of state varia les in linux kernel and Qualcomm MSM kernel (Google Pixel).
    \item \citetitle{ProFuzzBench}\cite{ProFuzzBench}: Suite of 10 protocols and 11 open-source implementations of those to be tested. TCP is notably absent from this list. Certain protocols (like FTP) already return HTTP status codes, others are patched to do so. Dockerized. The authors note that configuration is not taken into account and multi-party ($\geq 3$) protocols can not be fuzzed right now. Non-determinism in the programs make feedback (like code coverage) less predictable and thus fuzzing less performant because it introduces non-differentiable duplicate entries into the corpus. Speed is another issue, where complex setup-processes, costly network operations (resp. synchronization for me), and long multipart-inputs contribute. Finally, state identification is only superficially handled.
    \item \citetitle{AFLNET}\cite{AFLNET}: FTP and RTSP as targets, state transition (+coverage) feedback, corpus from traces, mutation on random entry (havoc and insert/delete/duplicate etc.), corpus scheduling based on statistics about each state.
    \item \citetitle{TCPFuzz}\cite{TCPFuzz}: TCP targets, differential fuzzing, input: syscalls + packets, complex mutators including intelligent dependencies between the current and all previous packets and syscalls in the input, feedback: inter-package diff in coverage.
    \item \citetitle{FitM}\cite{FitM}: Use persistent snapshots of userspace processes (CRIU), fuzzer-in-the-middle, multiple strategies such as input deduplication and resynchronization necessary because of the approach
    \item \citetitle{Autofuzz}\cite{Autofuzz}: Man-in-the-middle, learn protocol by constructing a FSA and packet syntax using a bioinformatics technique. Fuzz server and client.
    \item \citetitle{EPF}\cite{EPF}: Target SCADA, uses population-based simulated annealing to schedule which packet type to add/mutate next in conjunction with coverage feedback. Requires Scapy-compatible implementation of packet types to ensure packet structure.
    \item \citetitle{ModelBased}\cite{ModelBased}: Build FSM and minimize it, then use it to schedule basic mutations.
    \item \citetitle{GANFuzz}\cite{GANFuzz}: Learn GAN to generate next inputs. Targets Modbus-TCP.
    \item \citetitle{StatefulReview}\cite{StatefulReview}: Provides taxonomy of components and categorizes stateful fuzzers, compares approaches and lists challenges and future directions.
    \item \citetitle{Congestion}\cite{Congestion}: Manual FSM from RFC to generate abstract attack models against congestion control implementations, which are then transformed to concrete attacks strategies and tested on actual implementations.
    \item \citetitle{ModbusTCP}\cite{ModbusTCP} is a Modbus fuzzer, that only seems to use TCP as transport. Generation of packets happens only for Modbus itself. \citetitle{MTFStorm}\cite{MTFStorm} is follow-up work that does the same more systematically.
    \item \citetitle{IndustrialReview}\cite{IndustrialReview} review industrial control protocol. Notably, Modbus/TCP fuzzing seems common, but TCP is only used as transport layer, not target.

    \item \citetitle{MTA}\cite{MTA} use machine learning models to generate Modbus/TCP packets, but target Modbus and again only use TCP as a transportation layer.
    \item \citetitle{AnotherModbusTCP}\cite{AnotherModbusTCP} is another Modbus/TCP fuzzer that does not fuzz the TCP stack. See also\cite{ModbusTCP2}.
    \item A similar approach using a Scapy-based fuzzer was implemented for the industrial protocol EtherNet/IP, where TCP was declared as out-of-scope\cite{ENIP}.
    \item \citetitle{StateMachine}\cite{StateMachine} calculate a directed graph from the measured state variables, and schedule mutations based on a formula incorporating state depth, coverage, number of transitions, and number of mutations based on this state.
  \end{itemize}

  \subsubsection{TODO: Read}
  \begin{itemize}
    \item \citetitle{Survey}\cite{Survey}
  \end{itemize}

  \section{Implementation}

  \subsection{NativeSim}
  \subsection{Network Interfaces}
  \begin{itemize}
    \item Custom ethernet driver
    \item No changes to code, just added code, and includes in config/build system
    \item Shared memory layout/inner workings
    \item Manual Responses
  \end{itemize}

  \subsection{Coverage Information}
  \begin{itemize}
    \item LLVM pass: \code{trace-pc-guard}
    \item Added code/includes to build system
    \item Reset before input execution to improve reliability: Only code change
  \end{itemize}

  \subsection{State Inference Heuristic}
  \begin{itemize}
    \item Based on responses by server
    \item Parsing, different non-TCP responses classified
    \item On TCP packets, the header flags are used
  \end{itemize}

  \subsection{Input Modelling and Mutation}

  \subsubsection{Mutation Target}
  \begin{itemize}
    \item Random (Multipart)
    \item ReplayingStateful
    \item Seeding
  \end{itemize}

  \subsubsection{Input Modelling}
  \begin{itemize}
    \item Byte array vs. parsed structure
  \end{itemize}

  \subsubsection{Mutators}
  \begin{itemize}
    \item \code[Rust]{havoc_mutations()}
    \item TCP only, certain fixed values
    \item Full stack, certain fixed values
    \item Full stack, fixing mutators
    \item Appending
  \end{itemize}

  \subsection{Implementation Details}
  \subsubsection{LibAFL Structure}
  \begin{itemize}
    \item Input Scheduling Algorithm
    \item Mutation Scheduling Algorithm
    \item Oracles
    \item MapObserver/-Feedback
    \item Executor
  \end{itemize}

  \subsubsection{Helper Functionality}
  \begin{itemize}
    \item Manual connection
    \item PCAP
    \item Scripting
  \end{itemize}

  \subsection{LibAFL}
  \subsubsection{Overcommit}
  \subsubsection{Numeric Mutators}
  \subsubsection{Mapping Mutators}
  \subsubsection{TODO: others}

  \section{Results}
  \subsection{Throughput and Overcommit}

  \subsection{Input Modelling and Mutation}
  With fixed feedback:

  \begin{itemize}
    \item \code[Rust]{MultipartInput} and \code[Rust]{havoc_mutations()}
    \item \code[Rust]{ReplayingStatefulInput} and \code[Rust]{havoc_mutations()}
    \item \code[Rust]{ReplayingStatefulInput} and parsed structure with different mutators
  \end{itemize}

  \subsection{Feedback}
  \begin{itemize}
    \item Coverage (repeatability!)
    \item State marking
    \item State diff marking
  \end{itemize}

  \section{Discussion}

  \subsection{Feedback}
  \begin{itemize}
    \item How effective do different improvements seem to be
  \end{itemize}
  \subsection{Input Modelling and Mutation}
  \begin{itemize}
    \item How effective do different improvements seem to be
  \end{itemize}
  \subsection{Future Work}
  \subsubsection{Improved Performance}
  \begin{itemize}
    \item This was not prioritized further because memory was the limiting factor
    \item Semaphores instead of busy waiting
  \end{itemize}
  \subsubsection{Alternate Targets}
  \begin{itemize}
    \item Other OS network stacks
    \item Userland network stacks
  \end{itemize}
  \subsubsection{Improved Oracles}
  \begin{itemize}
    \item e.g. Differential Fuzzing
    \item Sanitizers
  \end{itemize}
  \subsubsection{Improved Scheduling Based on State Feedback}
  \subsubsection{Generilazation of Techniques}
  \subsubsection{Extended Evaluation}
  \begin{itemize}
    \item System calls
    \item IPv6
    \item Other network protocols
    \item Comparison to other fuzzers (not done because none were applicable without extensive engineering work to ensure compatibility with target)
  \end{itemize}

  \subsection{Contributions}
  \begin{itemize}
    \item Summary of paper
  \end{itemize}

\end{multicols}
\vspace{30px}
\begin{center}
  \begin{minipage}{0.65\textwidth}
    \centering
    In the interest of open science, the source code of this project is publicly available and released under an open-source license. During development, thousands of lines of code have been introduced to multiple upstream projects.

    All artifacts produced for this project are available at

    \vspace{8px}

    \href{https://github.com/riesentoaster/fuzzing-zephyr-network-stack}{github.com/riesentoaster/fuzzing-zephyr-network-stack}.
  \end{minipage}
\end{center}

\vspace{50px}

\begin{multicols}{2}
  \defbibheading{bibliography}[\bibname]{\section*{#1}}
  \addcontentsline{toc}{section}{\bibname}
  \printbibliography
\end{multicols}

\end{document}